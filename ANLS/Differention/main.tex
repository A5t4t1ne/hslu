\documentclass[12pt]{scrartcl}
\usepackage[ngerman]{babel}

\usepackage{amsmath, amssymb}

\usepackage{array}

\usepackage{nameref}

\usepackage{hyperref}


\hypersetup{
    colorlinks,
    citecolor=black,
    filecolor=black,
    linkcolor=black,
    urlcolor=black
}

\bibliographystyle{IEEetran}

\author{David Jäggli}

\title{Analysis Differential Rechnung}

\begin{document}

\maketitle

\tableofcontents

\newpage


\section{Ableitung} 
\subsection{Ziele der Ableitungen}

Die folgenden drei Punkte werden mittels eine Differentialrechnung erreicht.

\begin{enumerate}
    \item Das Bestimmen der Tangente in einem Punkt der Kurve $y=f(x)$
    \item Das Linearisieren einer Funktion $y=f(x)$ (siehe Kapitel \ref{approximation})
    \item Mit einer neuen Funktion $f'(x)$ (1. Ableitung von $f(x)$ die Steigung an jedem Punkt $x$ von der ursprünglichen Funktion $f(x)$ zu ermitteln.
    \item Mit einer neuen Funktion $f''(x)$ (2. Ableitung von $f(x)$) zu bestimmen ob sich die ursprüngliche Funktion $f(x)$ an jeder beliebigen Stelle $x$ in einer Links- resp. Rechtskrümmung befindet
\end{enumerate}


\section{Differenzierbarkeit}
Eine Funktion ist immer dann differenzierbar wenn sie stetig ist und keinen Knick 
hat.

\subsection{Regeln}
\subsubsection{Potenzregeln}

\renewcommand{\arraystretch}{1.5}
\begin{center}
    \begin{tabular}{ | m{10em} | m{10em} | m{10em} | }
        \hline
        Ursprungsfunktion & Ableitung & Beispiel \\ 
        \hline
        $f(x) = x^n$ & $f'(x) = n \cdot x^{(n-1)} $ & $x^2 \Rightarrow 2x$\\ 
        \hline
        $f(x) = c = 2$ (Konst.) & $f'(x) = 0$ & $5 \Rightarrow 0$ \\
        \hline
        $f(x) = e^x$ & $f'(x) = e^x$ & $e^x \Rightarrow e^x$ \\
        \hline
        $f(x) = a^x$ & $f'(x) = ln(a) \cdot a^x$ & $5^x \Rightarrow ln(5) \cdot 5^x$ \\
        \hline
        $f(x) = ln(x)$ & $f'(x) = \frac{1}{x}$ & $ln(x) \Rightarrow \frac{1}{x}$ \\
        \hline 
        $f(x) = \frac{a}{(x + n)} $ & $ f'(x) = -\frac{a}{(x+n)^2} $ & $\frac{500}{x + 5} \Rightarrow -\frac{500}{(x+5)^2}$\\
        \hline 
        $f(x) = log_a(x) $ & $f'(x) = \frac{1}{x \cdot ln(a)}$ & $log_3(x) \Rightarrow \frac{1}{x * ln(3)}$ \\
        \hline 

    \end{tabular}
\end{center}

% -------- begin of chain rules ----------------
\hspace{0pt} \\

\subsubsection{Regeln für verknüpfte Funktionen}

\begin{center}
\textbf{Faktorregel \& Beispiel:} 
\[ f(x) = a \cdot g(x) = a \cdot g'(x)\]
\[f(x)=2 \cdot x^3 \Rightarrow f'(x)=2 \cdot (x^3)' = 2 \cdot 3 \cdot x^2 = \underline{6x^2}\]

\newpage
\textbf{Summenregel \& Beispiel:} 
\[ (f(x) + g(x))' = f'(x) + g'(x) \]
\[ x^2 + x^3 \Rightarrow \underline{2x + 3x^2}\]

\hspace{0pt} \\
\hspace{0pt} \\
\noindent
\textbf{Produktregel \& Beispiel:} 
\[ (f(x) \cdot g(x))' =  f'(x) \cdot g(x) + f(x) \cdot g'(x)\]
\[x^3 \cdot x^3 \Rightarrow 3x^2 \cdot x^3 + x^3 \cdot 3x^2 = 3x^5 + 3x^5 = \underline{6x^5}\]


\hspace{0pt} \\
\hspace{0pt} \\
\noindent
\textbf{Quotientenregel \& Beispiel:} 
\[\left(\frac{f(x)}{g(x)}\right)' = \frac{f'(x) \cdot g(x) - f(x) \cdot g'(x)}{(g(x))^2}\]
\[f(x)= \frac{2x^5}{2x} \Rightarrow f'(x) = \frac{10x^4 \cdot 2x - 2x^5 \cdot 2}{(2x)^2} = \frac{20x^5 - 4x^5}{4x^2} = \frac{16x^5}{4x^2} = \underline{4x^3}\]

\hspace{0pt} \\
\hspace{0pt} \\
\noindent
\textbf{Kettenregel \& Beispiel:} 
\[f(g(x)) \Rightarrow (f(g(x)))' = f'(g(x)) \cdot g'(x)\]
\[f(x) = ln(5x^6) = ln(g(x)) \Rightarrow f'(x) = \frac{1}{g(x)}\]


\end{center}

% ------ End of chain rules ----------
\newpage

\section{Tangente \& Normale}
\subsection{Tangente einer Funktion}
Die Tangente an einem Punkt $x_0$ in einer Funktion lautet folgendermassen:\\
$ f'(x_0) \cdot (x - x_0) + y_0$ wobei $y_0 = f(x_0)$\\

\noindent
Der Winkel einer Geraden (Tangente) zur x-Achse kann mit der tangens resp arctan 
Funktion berechnet werden: $\alpha = arctan(f'(x_0)) \rightarrow$ an der Stelle 
$x_0$ hat die Tangente einen Winkel von $\alpha$ zur x-Achse.


\subsection{Normale}
Die Normale oder senkrechte Funktion lässt sich folgendermassen berechnen:\\
Aus  $g_1 \perp g_2  $ folgt, dass $m_1 \cdot m_2 = -1$ \\ 
resp. $m_1 = -\frac{1}{m_2}$ \\
\\ 
\noindent
Eine Gerade ist definiert durch: $y = mx + q$. Die Steigung $m$ der Normalen ist wie oben
beschrieben einfach der negative Kehrwert der Steigung der Tangente. Den $y$-Wert 
kann man durch Einsetzen des Punktes berechnen. \\

\noindent
\underline{\textbf{Beispiel:}} \\
Gegeben: $f(x) = x^2 -2x + 1 \quad \Rightarrow \quad f'(x) = 2x -2$\\
Gesucht: Tangente und Normale bei $x=0$\\
Steigung der Tangente: $m_1 = f'(x=0) = -2$ \\
Steigung der Normale: $m_2 = -\frac{1}{m_1} = 0.5$ \\
\\
Für q einfach die Steigung in der Gleichung einsetzen.\\
Durch $f(x)$ ist definiert, dass bei $x=0 \rightarrow y=1$, heisst:\\
Für die Tangente: $1 = m_1x + q \Rightarrow 1= -2x + q \Rightarrow q = 1$ weil $x=0$\\
Für die Normale: $1 = m_2x + q \Rightarrow 1= 0.5x + q \Rightarrow q = 1$ weil $x=0$\\
\\
\noindent
Tangente = \underline{$f(x) = -2x + 1$}\\
Normale = \underline{$f(x) = 0.5x + 1$}\\

\section{Annäherungen} \label{approximation}
Oftmals ist es nicht nötig (zu aufwändig/schwierig) oder gar nicht möglich die exakte 
Funktion zu berechnen (z.B. ln(x)). Um dieses Problem zu beheben, erstellt man eine Annäherungsfunktion
in einem gewissen Bereich, welche genügend nah an die ursprüngliche Funktion herankommt.


\subsection{Lineare Annäherungen}


Lineare Annäherung an einem Punkt $x_0$ ist die Tangente der 
Funktion am Punkt $x_0$

\[ \frac{f'( x_0 )}{1!} ( x - x_0) \Rightarrow f'( x_0 ) ( x - x_0) \] 

\subsection{Quadratische Annäherungen}

Quadratische Annäherung einer Funktion kommt viel näher an die tatsächliche Funktion heran, 
weil sie die Krümmung berücksichtigt. Eine Quadratische Annäherung kann mit folgender Formel
erreicht werden:
\[
    \frac{f^{(n)} ( x_0 )}{n!} ( x - x_0)^n +
    \frac{f^{(n-1)} ( x_0 )}{(n - 1)!} ( x - x_0)^{n - 1} + .... +
    \frac{f'' ( x_0 )}{2!} ( x - x_0)^2 + 
    \frac{f' ( x_0 )}{1} ( x - x_0)
\]

\noindent
Dabei muss aber beachtet werden, dass die Funktion $f(x)$ auch $n$-mal differenzierbar
ist.

% \bibliography{quantum_ready}

\end{document}