\documentclass[12pt]{scrartcl}
\usepackage[ngerman]{babel}


\usepackage{amsmath, amssymb}

\usepackage{array}  % for the tables

\usepackage{nameref}  % for referencing with name

\usepackage{hyperref}  % for hyperlinks

\usepackage{mathrsfs}

\usepackage{graphicx}  % for the images

\usepackage{xcolor, colortbl}

\usepackage{gensymb} % for \degree

\usepackage{pgfplots}

\usepackage{tabto}

\usetikzlibrary{arrows}

% \usepgfplotslibrary{external}

% \tikzexternalize

\definecolor{Gray}{gray}{0.85}

\setlength{\parindent}{0cm}

% hyperlinks
\hypersetup{
    colorlinks,
    citecolor=black,
    filecolor=black,
    linkcolor=black,
    urlcolor=black
}

\bibliographystyle{IEEetran}




\author{David Jäggli}

\title{Diskrete Mathematik - Übungen SW01}



% ---------- Begin Main Document ----------- %



\begin{document}

\maketitle

\tableofcontents

\newpage
\section{Logik}
\textbf{1.)}
\begin{center}
    \begin{tabular}{ c | c | c | c | c | c | c | c }
        $p$ & $q$ & $r$ & $p \land q$ & $q \lor r$ & $p \land (q \lor r)$ & $p \land r$ & $(p \land q) \lor (p \land r)$\\ 
        \hline
        w & w & w & w & w & w & w & w \\ 
        w & w & f & w & w & w & f & w \\ 
        w & f & w & f & w & w & w & w \\ 
        w & f & f & f & f & f & f & f \\ 
        f & w & w & f & w & f & f & f \\ 
        f & w & f & f & w & f & f & f \\ 
        f & f & w & f & w & f & f & f \\ 
        f & f & f & f & f & f & f & f \\ 
    \end{tabular}
\end{center}

Letzte Spalte entspricht der 3. letzten Spalte, wodurch 
$p \land (q \lor r) \equiv (p \land q) \lor (p \land r)$
belegt wurde.

\vspace{50px}
\textbf{I.)}
\begin{center}
    \begin{tabular}{ c | c | c | c }
        $p$ & $q$ & $p \land q$ & $p \lor (p \land q)$\\ 
        \hline
        w & w & w & w \\ 
        w & f & f & w \\ 
        f & w & f & f \\ 
        f & f & f & f \\ 
    \end{tabular}
\end{center}

Die letzte Spalte entspricht der ersten Spalte, wodurch 
$p \lor (p \land q) \equiv p$ 
belegt wurde.

\vspace{50px}
\textbf{II.)}
\begin{center}
    \begin{tabular}{ c | c | c | c | c | c | c | c }
        $p$ & $q$ & $r$ & $p \rightarrow q$ & $q \rightarrow r$ & $p \rightarrow$ r & $(p \rightarrow q) \land (q \rightarrow r)$ & $(p \rightarrow q) \land (q \rightarrow r) \rightarrow (p \rightarrow r)$\\ 
        \hline
        w & w & w & w & w & w & w & w \\ 
        w & w & f & w & f & f & f & w \\ 
        w & f & w & f & w & w & f & w \\ 
        w & f & f & f & w & f & f & w \\ 
        f & w & w & w & w & w & w & w \\ 
        f & w & f & w & f & w & f & w \\ 
        f & f & w & w & w & w & w & w \\ 
        f & f & f & w & w & w & w & w \\ 
    \end{tabular}
\end{center}

Die letzte Spalte ist in allen Fällen wahr, wodurch belegt wurde, dass die
Aussage eine Tautologie ist.


\textbf{III.)}\\
$(p \rightarrow q) \land ( q \rightarrow r) \rightarrow (p \rightarrow r)$\\
$\equiv (\lnot p \lor q) \land (\lnot q \lor r) \rightarrow (\lnot p \lor r)$\\
$\equiv \lnot((\lnot p \lor q) \land (\lnot q \lor r)) \lor \lnot p \lor r$\\
$\equiv \lnot (\lnot p \lor q) \lor \lnot (\lnot q \lor r) \lor \lnot p \lor r$\\
$\equiv p \lor \lnot q \lor q \lor \lnot r \lor \lnot p \lor r$\\
$\equiv p \lor \lnot p \lor q \lor \lnot q \lor r \lor \lnot r $\\
$\equiv \mathbf{T} \lor \mathbf{T} \lor \mathbf{T}$\\
$\equiv \mathbf{T}$


\vspace{50px}
\textbf{IV.)}
\begin{center}
    \begin{tabular}{ c | c | c | c | c | c | c}
        $p$ & $q$ & $\lnot p$ & $\lnot q$ & $p \rightarrow q$ & $p \land (p \rightarrow q)$ & ($p \land (p \rightarrow q)) \rightarrow \lnot p$\\ 
        \hline
        w & w & f & f & w & w & w \\ 
        w & f & f & w & f & f & w \\ 
        f & w & w & f & w & f & w \\ 
        f & f & w & w & w & f & w \\ 
    \end{tabular}
\end{center}

Die letzte Spalte ist immer wahr wodurch mit der Tabelle belegt wurde,
dass der Ausdruck eine Tautologie ist. \\

$(\lnot q \land (p \rightarrow q)) \rightarrow \lnot p$\\
$\equiv (\lnot q \land (\lnot p \lor q)) \rightarrow \lnot p$\\
$\equiv (\lnot q \land \lnot p \lor \lnot q \land q) \rightarrow \lnot p$\\
$\equiv (\lnot q \land \lnot p) \rightarrow \lnot p$\\
$\equiv \lnot(q \lor p) \rightarrow \lnot p$\\
$\equiv q \lor p \lor \lnot p$\\
$\equiv q \lor \mathbb{T}$\\
$\equiv \mathbb{T}$\\

\newpage
\section{Prädikate und Quantoren}


\textbf{V.)}\\
(a). wahr\\
(b). wahr\\
(c). falsch\\       
(d). falsch\\
Begründung:\\
a. ist so weil ist so\\
b. bei $n = \frac{2}{3}$ stimmt der Ausdruck $\rightarrow$ Aussage ist wahr\\
c. Keine Zahl entspricht ihrem negativen äquivalent\\
d. $0.5^2 = 0.25$ $\rightarrow$ Aussage ist falsch.\\


Korrektur:
\begin{itemize}
    \item bei c(n=0) ist stimmt n = -n $\rightarrow$ Aussage ist wahr
    \item bei d habe ich überlesen, dass $n \in \mathbb{Z}$ sein muss
\end{itemize}


\section{Beweise}

\textbf{VI.)}\\
Wenn diese Aussage wahr ist dann dürfen in keinem Monat mehr als 2 Tage sein.
Falls in jedem Monat 2 Meetings stattfinden kämen wir auf 24 Meetings. Das letzte muss aber
auch noch irgendwo hin $\rightarrow$ in einem Monat sind 3 Meetings.


\vspace{50px}
\textbf{VII.)}\\
Ke blasse wimer da hätt uf die Lösig söue cho. Isch aber no gschid.

% \renewcommand{\arraystretch}{1.5}
% \begin{center}
%     \begin{tabular}{ | m{12em} | m{12em} | m{12em} | }
%         \hline
%         1 & 2 & 3\\ 
%         \hline
%         1 & 2 & 3\\ 
%         \hline
%         1 & 2 & 3\\ 
%         \hline
%     \end{tabular}
% \end{center}


% tabular example 2 columns
% \renewcommand{\arraystretch}{1.5}
% \begin{center}
%     \begin{tabular}{ | m{17em} | m{17em} | }
%         \hline
%         1 & 2\\ 
%         \hline
%         1 & 2\\ 
%         \hline
%         1 & 2\\ 
%         \hline
%     \end{tabular}
% \end{center}


% \begin{tikzpicture}[line cap=round,line join=round,>=triangle 45,x=0.5cm,y=0.25cm]
%     \begin{axis}[
%     x=0.75cm,y=0.5cm, % size of the grid
%     axis lines=middle,
%     ymajorgrids=true,
%     xmajorgrids=true,
%     xmin=-10,
%     xmax=10,
%     ymin=-10,
%     ymax=10,
%     xtick={-11,-10,...,10},
%     ytick={-10,-9,...,9},]
%     \draw[line width=2pt,color=blue] (-10,-5) -- (-2,-1);
%     \begin{scriptsize}
%         \draw[color=blue] (-9.866,-4.728) node {$g$};
%         \draw[color=blue] (-1.906,7.172) node {$f$};
%         \draw[color=blue] (3.134,5.232) node {$h$};
%     \end{scriptsize}
% \end{axis}
% \end{tikzpicture}




% \bibliography{quantum_ready}

\end{document}