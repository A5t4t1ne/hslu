\documentclass[12pt]{scrartcl}
\usepackage[ngerman]{babel}


\usepackage{amsmath, amssymb}

\usepackage{array}  % for the tables

\usepackage{nameref}  % for referencing with name

\usepackage{hyperref}  % for hyperlinks

\usepackage{mathrsfs}

\usepackage{graphicx}  % for the images

\usepackage{xcolor, colortbl}

\usepackage{gensymb} % for \degree

\usepackage{pgfplots}

\usepackage{tabto}

\usepackage{ulem} % \uuline

\usetikzlibrary{arrows}

% \usepgfplotslibrary{external}

% \tikzexternalize

\definecolor{Gray}{gray}{0.85}

\setlength{\parindent}{0cm}

\newcommand{\RomanNumeralCaps}[1]
    {\MakeUppercase{\romannumeral #1}}

% hyperlinks
\hypersetup{
    colorlinks,
    citecolor=black,
    filecolor=black,
    linkcolor=black,
    urlcolor=black
}

\bibliographystyle{IEEetran}




\author{David Jäggli}

\title{Diskrete Mathematik - Übungen SW02}



% ---------- Begin Main Document ----------- %



\begin{document}

\maketitle

\tableofcontents

\newpage
\section{Funktionen}
\textbf{I.)}\\
$\mathbb{D} = \mathbb{R}$\\
$\mathbb{W} = \mathbb{N}$\\

\textbf{Korrektur I.)}\\
$\mathbb{D} =$ alle binäre Strings\\
$\mathbb{W} = 2\mathbb{N}$\\


\textbf{II.)}\\
Gegeben:\\
$f(x) = x^2 + 1$\\
$g(x) = x + 2$\\
whereas $\mathbb{R} \mapsto \mathbb{R}$\\

Gesucht:
\begin{enumerate}
    \item $f \circ g$
    \item $g \circ f$
    \item $f + g$
    \item $f \cdot g$\\
\end{enumerate}

$f \circ g = f(x + 2) = (x + 2)^2 + 1 =$ \underline{$x^2 + 4x + 5$}\\
$g \circ f = g(x^2 + 1) = (x^2 + 1) + 2 =$ \underline{$x^2 + 3$}\\
$f + g = (x^2 + 1) + (x + 2) =$ \underline{$x^2 + x + 3$}\\
$f \cdot g = (x^2 + 1) \cdot (x + 2) =$ \underline{$x^3 + 2x^2 + x + 2$}\\


\newpage
\section{Folgen, Reihen, Summen- und Produktzeichen}
\textbf{III.)}\\
a)
$\displaystyle{\prod_{i=0}^{10} i} = 0$\\

b)
$\displaystyle{\prod_{i=5}^{8} i} = 1680$\\

c)
$\displaystyle{\prod_{i=1}^{100} (-1)^i} = -1$\\

d)
$\displaystyle{\prod_{i=1}^{10} 2} = 2^{10}$\\

\textbf{Korrektur:}\\
c) anscheinend gilt $\displaystyle{\prod_{i=1}^{n} (-1)^i = (-1)^\frac{n \cdot (n+1)}{2}}$ wodurch\\
$\displaystyle{\prod_{i=1}^{100} (-1)^i = (-1)^\frac{100 \cdot 101}{2}} = (-1)^{5050} = 1$


\newpage
\textbf{IV.)}\\
a)
$\displaystyle{\sum_{j=0}^{8} 3 \cdot 2^j} = 3 \cdot (1 + 2 + 4 + 8 + 16 + 32 + 64 + 128 + 256) = 3 \cdot (2^9 - 1) = 1533$\\

b)
$\displaystyle{\sum_{k=1}^{8} 2^k} = 2^9 - 2 = 510$\\

c)
$\displaystyle{\sum_{l=2}^{8} (-3)^l} = (-3)^2 + (-3)^3 + (-3)^4 + (-3)^5 + (-3)^6 + (-3)^7 + (-3)^8 = 4923$\\

d)
$\displaystyle{\sum_{i=0}^{8} 2 \cdot (-3)^i} = 2 \cdot (c + 1 - 3) = 9842$\\


\textbf{Korrektur:}\\

Forgot about $\sum_{k=0}^{n} x^k = \frac{x^{n+1} - 1}{x - 1}$

\newpage
\textbf{V.)}\\

$\displaystyle{\sum_{k=99}^{200} k^3} = \sum_{k=1}^{200} k^3 - \sum_{k=1}^{98} k^3 = \frac{200^2 \cdot 201^2}{4} - \frac{98^2 \cdot 99^2}{4} = 380477799$\\
\\
\\


\textbf{VI.)}\\

a) $\displaystyle{\sum_{j=0}^{4} j! = 1 + 1 + 2 + 6 + 24 = 34}$

b) $\displaystyle{\prod_{j=0}^{4} j! = 1 \cdot 1 \cdot 2 \cdot 6 \cdot 24 = 288 }$\\
\\
\\


\textbf{VII.)}\\

\[\sum_{j=1}^n (a_j - a_{j-1}) = a_n - a_0\]
$\sum_{j=1}^n (a_j - a_{j-1}) = (a_1 - a_0) + (a_2 - a_1) \dots (a_{n-1} - a_{n-2}) + (a_n - a_{n-1}) \\
= a_1 - a_0 + a_2 - a_1 + \dots + a_{n-1} - a_{n-2} + a_n - a_{n-1}\\
= -a_0 + a_1 - a_1 + a_2 - a_2 + \dots + a_{n-2} - a_{n-2} + a_{n-1} - a_{n-1} + a_n\\
= a_n - a_0$


% tabular example 3 columns
% \renewcommand{\arraystretch}{1.5}
% \begin{center}
%     \begin{tabular}{ | m{12em} | m{12em} | m{12em} | }
%         \hline
%         1 & 2 & 3\\ 
%         \hline
%         1 & 2 & 3\\ 
%         \hline
%         1 & 2 & 3\\ 
%         \hline
%     \end{tabular}
% \end{center}


% tabular example 2 columns
% \renewcommand{\arraystretch}{1.5}
% \begin{center}
%     \begin{tabular}{ | m{17em} | m{17em} | }
%         \hline
%         1 & 2\\ 
%         \hline
%         1 & 2\\ 
%         \hline
%         1 & 2\\ 
%         \hline
%     \end{tabular}
% \end{center}

% \bibliography{}

\end{document}