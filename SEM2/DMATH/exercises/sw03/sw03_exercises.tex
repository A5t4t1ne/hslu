\documentclass[12pt]{scrartcl}
\usepackage[ngerman]{babel}


\usepackage{amsmath, amssymb}

\usepackage{array}  % for the tables

\usepackage{nameref}  % for referencing with name

\usepackage{hyperref}  % for hyperlinks

\usepackage{mathrsfs}

\usepackage{graphicx}  % for the images

\usepackage{xcolor, colortbl}

\usepackage{gensymb} % for \degree

\usepackage{pgfplots}

\usepackage{tabto}

\usepackage{ulem} % \uuline

\usetikzlibrary{arrows}

% \usepgfplotslibrary{external}

% \tikzexternalize

\definecolor{Gray}{gray}{0.85}

\setlength{\parindent}{0cm}

\newcommand{\RomanNumeralCaps}[1]
    {\MakeUppercase{\romannumeral #1}}

% hyperlinks
\hypersetup{
    colorlinks,
    citecolor=black,
    filecolor=black,
    linkcolor=black,
    urlcolor=black
}

\bibliographystyle{IEEetran}




\author{David Jäggli}

\title{Diskrete Mathematik - Übungen SW03}



% ---------- Begin Main Document ----------- %



\begin{document}

\maketitle

\tableofcontents

\newpage
\section{Big-$\mathcal{O}$ Notation}
\textbf{I.)}\\
a.)\\
$n \log(n^2+1) + n^2 \log n \leq n^2 \log(n^2 + 1) + n^2 \log(n^2+1) \quad\quad \forall n \geq 1$\\
$\leq 2n^2 \log(n^2 + 1) \quad\quad \forall n \geq 1$\\
$\leq 2n^2 \log(n^3) \quad\quad \forall n \geq 1$\\
$\leq 6n^2 \log(n) \quad\quad \forall n \geq 1$\\

Also $n \log(n^2+1) + n^2 \log n \in \mathcal{O}(n^2 \log(n))$ mit den Zeugen $C=6$ und $k=1$.\\

\textbf{Korrektur:}\\
k = 2, wegen $x^2 + 1 \leq n^3$, nur wenn $n \geq 2$.\\


b.)\\
Ganz ehrlich: keine Ahnung, bin auch mit dem Tip auszumultiplizieren nicht aufs richtige Ergebniss
gekommen.\\


c.)\\
$n^{2^n} + n^{n^2} \leq n^{2^n} + n^{2^n} \quad\quad \forall n \geq 4$\\
$\leq  2n^{2^n}$\\

Heisst: $n^{2^n} + n^{n^2} \in \mathcal{O}(n^{2^n})$ mit $C = 2$ und $k=4$


\section{Zahlen und Divisionen}
\textbf{II.)}\\

ggT(12345, 54321)=\\

$54321 = 12345 \cdot 5 + 4941$\\
$12345 = 4941 \cdot 2 + 2463$\\
$4941 = 2463 \cdot 2 15$\\
$2463 = 15 \cdot 145 + 3$\\
$15 = 3 \cdot 5 + 0 $\\

ggT(12345, 54321) = 3\\


\newpage
\section{Matrizen}
\textbf{III.)}\\

$\begin{bmatrix}
2 & 3 \\
1 & 4 \\
\end{bmatrix} \cdot
$
$
\begin{bmatrix}
a_{1,1} & a_{1,2} \\
a_{2,1} & a_{2,2} \\
\end{bmatrix} =
$
$
\begin{bmatrix}
3 & 0 \\
1 & 2 \\
\end{bmatrix}
$\\

$2 \cdot a_{1,1} + 3 \cdot a_{1, 2} = 3$\\
$1 \cdot a_{1,1} + 4 \cdot a_{1, 2} = 1$\\
$2 \cdot a_{1,2} + 3 \cdot a_{2, 2} = 0$\\
$1 \cdot a_{2,1} + 4 \cdot a_{2, 2} = 2$\\


4 Unbekannte \& 4 Gleichungen $\rightarrow$ Gleichung auflösen, daraus ergibt sich:\\
$a_{1,1} = \frac{9}{5}$\\
$a_{1,2} = -\frac{1}{5}$\\
$a_{2,1} = -\frac{6}{5}$\\
$a_{2,2} = \frac{4}{5}$\\




\textbf{IV.)}\\
$\mathbf{A} = 
\begin{bmatrix}
1 & 0 & 1 \\
1 & 1 & 0 \\
0 & 0 & 1 \\
\end{bmatrix} \quad
$ und
$\quad \mathbf{B} = 
\begin{bmatrix}
0 & 1 & 1 \\
1 & 0 & 1 \\
1 & 0 & 1 \\
\end{bmatrix}
$\\


$\mathbf{A} \lor \mathbf{B} =
\begin{bmatrix}
    1 & 1 & 1 \\
    1 & 1 & 1 \\
    1 & 0 & 1 \\
\end{bmatrix}
$,
$\quad \mathbf{A} \land \mathbf{B} =
\begin{bmatrix}
    0 & 0 & 1 \\
    1 & 0 & 0 \\
    0 & 0 & 1 \\
\end{bmatrix}
$,
$\quad \mathbf{A} \odot \mathbf{B} =
\begin{bmatrix}
    0 & 1 & 1 \\
    1 & 0 & 1 \\
    1 & 0 & 1 \\
\end{bmatrix}
$\\

\vspace{25px}

\textbf{Korrektur:}\\
$\mathbf{A} \odot \mathbf{B} =
\begin{bmatrix}
    1 & 1 & 1 \\
    1 & 1 & 1 \\
    1 & 0 & 1 \\
\end{bmatrix}
$

% tabular example 3 columns
% \renewcommand{\arraystretch}{1.5}
% \begin{center}
%     \begin{tabular}{ | m{12em} | m{12em} | m{12em} | }
%         \hline
%         1 & 2 & 3\\ 
%         \hline
%         1 & 2 & 3\\ 
%         \hline
%         1 & 2 & 3\\ 
%         \hline
%     \end{tabular}
% \end{center}


% tabular example 2 columns
% \renewcommand{\arraystretch}{1.5}
% \begin{center}
%     \begin{tabular}{ | m{17em} | m{17em} | }
%         \hline
%         1 & 2\\ 
%         \hline
%         1 & 2\\ 
%         \hline
%         1 & 2\\ 
%         \hline
%     \end{tabular}
% \end{center}

% \bibliography{}

\end{document}