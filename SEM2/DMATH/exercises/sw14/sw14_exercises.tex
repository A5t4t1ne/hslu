\documentclass[12pt]{scrartcl}
\usepackage[ngerman]{babel}


\usepackage{amsmath, amssymb}

\usepackage{array}  % for the tables

\usepackage{nameref}  % for referencing with name

\usepackage{hyperref}  % for hyperlinks

\usepackage{mathrsfs}

\usepackage{graphicx}  % for the images

\usepackage{xcolor, colortbl}

\usepackage{gensymb} % for \degree

\usepackage{pgfplots}

\usepackage{tabto}

\usepackage{ulem} % \uuline

\usetikzlibrary{arrows}

% \usepgfplotslibrary{external}

% \tikzexternalize

\definecolor{Gray}{gray}{0.85}

\setlength{\parindent}{0cm}

\newcolumntype{C}[1]{>{\centering\let\newline\\\arraybackslash\hspace{0pt}}m{#1}}

\newcommand{\RomanNumeralCaps}[1]
    {\MakeUppercase{\romannumeral #1}}

% hyperlinks
\hypersetup{
    colorlinks,
    citecolor=black,
    filecolor=black,
    linkcolor=black,
    urlcolor=black
}

\bibliographystyle{IEEetran}




\author{David Jäggli}

\title{Diskrete Mathematik - Übungen SW14}



% ---------- Begin Main Document ----------- %


\begin{document}

\maketitle

\tableofcontents

\newpage
\section{Einführung in die Graphentheorie III}
\textbf{I}\\
Dijkstra-Algorithmus\\
\renewcommand{\arraystretch}{1.5}
\begin{center}
    \begin{tabular}{ | C{3em} | C{3em} | C{3em} | C{3em} | C{3em} | C{3em} | C{3em} | C{3em} | C{3em} | }
        \hline
        L(a) & 0        &           &           &           &           &           &           & \\
        \hline
        L(b) & $\infty$ & 1         &           &           &           &           &           & \\
        p(b) &          & a         &           &           &           &           &           & \\
        \hline
        L(c) & $\infty$ & $\infty$  & 8         & 8         & 5         & 5         & & \\
        p(c) &          &           & b         & b         & d         & d         & & \\
        \hline
        L(d) & $\infty$ & 4         & 3         & 3         &           &           &           & \\
        p(d) &          & a         & b         & b         &           &           &           & \\
        \hline
        L(e) & $\infty$ & $\infty$  & $\infty$  & 3         & 3         &           & & \\
        p(e) &          &           &           & g         & g         &           & & \\
        \hline
        L(f) & $\infty$ & $\infty$  & $\infty$  & $\infty$  & $\infty$  & 5         & 5         & \\
        p(f) &          &           &           &           &           & e         & e         & \\
        \hline
        L(g) & $\infty$ & 2         & 2         &           &           &           &           & \\
        p(g) &          & a         & a         &           &           &           &           & \\
        \hline
        L(h) & $\infty$ & 6         & 6         & 5         & 5         & 5         & 5         & 5 \\
        p(h) &          & a         & a         & g         & g         & g         & g         & g \\
        \hline
        S & a & b & g & d & e & c & f & h \\
        \hline
    \end{tabular}
\end{center}


\newpage
\



% \begin{align*}
%     1 &= 1 \\
% \end{align*}


% Abschnittsweise definierte Funktionen
% \[ y = g(x) = 
%     \begin{cases} 
%     \frac{1}{2}x    & x \in ]-\infty; -2] \\
%     -2x+3           & x \in ]-2; 3]\\
%     5               & x \in ]3;\infty[
%  \end{cases}
% \]

% Matrix example
% \textbf{Korrektur:}\\
% $\mathbf{A} \odot \mathbf{B} =
% \begin{bmatrix}
%     1 & 1 & 1 \\
%     1 & 1 & 1 \\
%     1 & 0 & 1 \\
% \end{bmatrix}
% $

% tabular example 3 columns
% \renewcommand{\arraystretch}{1.5}
% \begin{center}
%     \begin{tabular}{ | m{12em} | m{12em} | m{12em} | }
%         \hline
%         1 & 2 & 3\\ 
%         \hline
%         1 & 2 & 3\\ 
%         \hline
%         1 & 2 & 3\\ 
%         \hline
%     \end{tabular}
% \end{center}


% tabular example 2 columns
% \renewcommand{\arraystretch}{1.5}
% \begin{center}
%     \begin{tabular}{ | m{17em} | m{17em} | }
%         \hline
%         1 & 2\\ 
%         \hline
%         1 & 2\\ 
%         \hline
%         1 & 2\\ 
%         \hline
%     \end{tabular}
% \end{center}

% \bibliography{}

\end{document}