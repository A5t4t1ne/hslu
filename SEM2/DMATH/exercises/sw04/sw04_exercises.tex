\documentclass[12pt]{scrartcl}
\usepackage[ngerman]{babel}


\usepackage{amsmath, amssymb}

\usepackage{array}  % for the tables

\usepackage{nameref}  % for referencing with name

\usepackage{hyperref}  % for hyperlinks

\usepackage{mathrsfs}

\usepackage{graphicx}  % for the images

\usepackage{xcolor, colortbl}

\usepackage{gensymb} % for \degree

\usepackage{pgfplots}

\usepackage{tabto}

\usepackage{ulem} % \uuline

\usetikzlibrary{arrows}

% \usepgfplotslibrary{external}

% \tikzexternalize

\definecolor{Gray}{gray}{0.85}

\setlength{\parindent}{0cm}

\newcommand{\RomanNumeralCaps}[1]
    {\MakeUppercase{\romannumeral #1}}

% hyperlinks
\hypersetup{
    colorlinks,
    citecolor=black,
    filecolor=black,
    linkcolor=black,
    urlcolor=black
}

\bibliographystyle{IEEetran}




\author{David Jäggli}

\title{Diskrete Mathematik - Übungen SW04}



% ---------- Begin Main Document ----------- %



\begin{document}

\maketitle

\tableofcontents

\newpage
\section{Vollständige Induktion}
\textbf{I.)}\\
Folgende Aussage beweisen: $H_{2n} \geq 1+\displaystyle{\frac{n}{2}}$, $\quad\forall n \in \mathbb{N}$.\\

1. \textbf{IV:} für $n=0$: $\quad1 \geq 1 = $ wahr.\\
2. \textbf{IS:} \\
\[H_{2^{n+1}} \geq 1 + \frac{n}{2} = 1 + \frac{1}{4} + \frac{1}{8} + \dots + \frac{1}{2^n} + \frac{1}{2^{n+1}} = H_{2^n} + \frac{1}{2^{n+1}}\] 

{\huge \color{red}{$\times$}\par}

Keine Ahnung warum $H_{2^{n+1}}$ so definiert ist, wie in den Lösungen.\\


\textbf{II.)}\\

\[\binom{a}{k} =
\begin{cases}
    \frac{a \cdot (a-1) \cdot \dots \cdot (a-k-1)}{k!} & k>0\\
    1 & k=0
\end{cases}
\]

$\displaystyle{\binom{2n+1}{n} > 2^{n+1}}$\\

\textbf{Induktionsverankerung:}\\

$a=2$, $\quad\displaystyle{\binom{4+1}{2} > 2^{2+1} = \frac{5 \cdot 4 \cdot 3 \cdot 2}{2}} > 8$\\
$a=3$, $\quad\displaystyle{\binom{6+1}{3} > 2^{3+1} = \frac{7 \cdot 6 \cdot 5 \cdot 4 \cdot 3   }{6}} > 16$\\

Alle wahr\\

\textbf{Induktionsschritt:}\\

$\quad\displaystyle{\binom{2(n+1)+1}{n+1} = \binom{2n+3}{n+1} > 2^{n+2}}$\\

\[\binom{2n+3}{n+1} = \frac{(2n+3) \cdot (2n+2) \cdot (n+3) \dots (n+1) \cdot 1}{1 \cdot 2 \cdot 3 \dots n \cdot (n+1)}\]

Ganz ehrlich: wie man von hier auf die richtige Lösung kommt, ist absolut random.

\newpage
\textbf{III.)}\\

Es gilt folgendes zu beweisen:\\

$\displaystyle{\sum_{k=0}^{n} (2k+1)^2 = \frac{(n+1)(2n+1)(2n+3)}{3}}$, $\quad n \in \mathbb{N}$\\

\textbf{Induktionsverankerung:}\\

$\displaystyle{1^2 = \frac{1 \cdot 1 \cdot 3}{3}} = $ true\\

$\displaystyle{1^2 + 3^2 = \frac{2 \cdot 3 \cdot 5}{3}} = $ true\\


\textbf{Induktionsschritt:}\\

$\displaystyle{1^2 + 3^2 + 5^2 + \dots + (2n + 1)^2 + (2n + 3)^2 = \frac{(n+1)(2n+1)(2n+3)}{3} + \frac{3 \cdot (2n + 3)^2}{3}} = $\\

$\displaystyle{\frac{4n^3 + 12n^2 + 11n + 3 + 12n^2 + 36n + 27}{3} = \frac{4n^3 + 24n^2 + 47n + 30}{3}} = $\\

$\displaystyle{\frac{(n+2)(2n+3)(2n+5)}{3}}$



\newpage
\section{Rekursiv definierte Funktionen}
\textbf{V.)}\\

$H1: \forall x [f(x) \rightarrow r(x) \lor s(x)]$\\
$H2: f(DI) \lor f(DO)$\\
$H3: \lnot (r(DI) \lor s(DI)))$\\
$H4: \lnot s(DO)$\\

H1: $f(DI) \rightarrow (r(DI) \lor s(DI))$ zusammen mit H3 = $\lnot f(DI)$\\
Dann muss H2: $f(DO)$\\

H1: $f(DO) \rightarrow r(DO) \lor s(DO)$ zusammen mit H4 und obigem Ergebniss: $r(DO)$\\

Ergibt gesamthaft:
\begin{enumerate}
    \item $\lnot f(DI)$: Dienstag regnet es nicht
    \item $f(DO)$: Ich nehme am Donnerstag frei
    \item $r(DO)$: Es regnet am Donnerstag
\end{enumerate}


% Matrix example
% \textbf{Korrektur:}\\
% $\mathbf{A} \odot \mathbf{B} =
% \begin{bmatrix}
%     1 & 1 & 1 \\
%     1 & 1 & 1 \\
%     1 & 0 & 1 \\
% \end{bmatrix}
% $

% tabular example 3 columns
% \renewcommand{\arraystretch}{1.5}
% \begin{center}
%     \begin{tabular}{ | m{12em} | m{12em} | m{12em} | }
%         \hline
%         1 & 2 & 3\\ 
%         \hline
%         1 & 2 & 3\\ 
%         \hline
%         1 & 2 & 3\\ 
%         \hline
%     \end{tabular}
% \end{center}


% tabular example 2 columns
% \renewcommand{\arraystretch}{1.5}
% \begin{center}
%     \begin{tabular}{ | m{17em} | m{17em} | }
%         \hline
%         1 & 2\\ 
%         \hline
%         1 & 2\\ 
%         \hline
%         1 & 2\\ 
%         \hline
%     \end{tabular}
% \end{center}

% \bibliography{}

\end{document}